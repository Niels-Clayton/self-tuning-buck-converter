\chapter{Evaluation}\label{C:evaluation}

This evaluation will look to evaluate the the success of the designed and implemented components, in achieving their specified requirements as outlined in \Cref{C:design}.

% The evaluation of the system will be conducted using a range of load resistances, evaluating its performance with 10$\Omega$, 15$\Omega$, and 20$\Omega$ output loads, using a constant supply voltage of 12V DC.
% S
% \begin{itemize}
% 	\item The buck converter will be able to take input voltages up to 12V DC
	
% 	\item The buck converter must maintain the basic functionality outlined in \cref{E:V_out}

% 	\item The buck converter will have an output voltage range between 3V and 10V DC
	
% 	\item The output voltage accuracy will be within $\pm 5\%$ of the target output voltage

% 	\item The user will be able to define the inductor ripple between 20\% and 50\%
	
% 	\item The inductor ripple accuracy will be within $\pm 5\%$ of the defined inductor ripple
	
% 	\item The buck converter will have a switching frequency range of 1k$Hz$ to 100k$Hz$

% 	\item The control system will have no steady state error

% \end{itemize}


\section{PWM Generation}\label{S:pwm_gen_eval}

\subsection*{Duty Cycle Variation}

\subsection*{Switching Frequency Variation}

\section{System State Sensing}\label{S:current_sense_eval}

\subsection{Output Voltage Sensing}

\subsection{Inductor Current Sensing}

\subsection*{Average Inductor Current Sensing}

\subsection*{Peak Inductor Current Sensing}

\section{Control System}\label{S:control_eval}

\subsection*{Output Voltage Control}

