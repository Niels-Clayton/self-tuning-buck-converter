\chapter{Introduction}\label{C:intro}

Historically power distribution has been primarily in the form of AC (Alternating current). This is credited to the fact that AC power is more efficient to transmit over long distances and made it easy to step up and down the voltages efficiently with transformers \cite{Earley2013}. However, with the invention of solid-state electronics such as the metal–oxide–semiconductor field-effect transistor or MOSFET for short, it has become possible to efficiently step up and step down direct current (DC) voltages. This has been achieved by the invention of the switch-mode power supply, which has facilitated the continued reduction of size and increase in efficiency of electronics \cite{Bocock}.

Today, switch-mode power supplies can be found in a wide variety of consumer and professional electronics, with some examples being laptops, phones, and any form of DC charger. Their widespread usage when compared to other DC-DC converters such as linear regulators can be attributed to their far greater efficiency. One such switch-mode power supply is the buck converter, which will step down a DC input voltage to a lower DC output voltage.

\section{Project Motivation}

Although buck converters are a widespread technology, they are not without their limitations and drawbacks. The current buck converter design process requires that a specific output filter be designed around the switching frequency of the converter, and the desired inductor ripple. This filter design process will often result in the converter requiring discrete passive components that are non-standard and hard to source. This will usually result in the designer having to make compromises in their design for either the cost or the performance of the converter.

Another drawback of this design process is the static nature of both the filter and the switching components once they have been selected. This results in the buck converters desired inductor current ripple only being achieved at a very specific designed output voltage or load. This means that with current buck converter designs varying the desired output voltage or varying the output load will cause the inductor current ripple to vary. This is an issue, as very few loads are static and will not change during their operation. 

\section{Project Goals}

This project aims to eliminate the need to design the output stage of a buck converter. By implementing a control system that varies the switching frequency of the converter, we will be able to directly manipulate the inductor current ripple. This project aims to produce a proof of concept buck converter that is capable of operating at 12V, with an output range of 3-10V and precision of $\pm5\%$. The converter will also be able vary its switching frequency between 1$kHz$ and 100$kHz$, allowing for selection of inductor current ripple between 20\% and 50\% with precision of $\pm5\%$. All of this must be implemented while maintaining the standard functionality of the converter.

\subsubsection{TODO Expand on the project goals, possibly convert the previous section into a bulleted list of project goals and requirements.}

This will be referenced in the implementation and design sections many times, so it needs to clearly outline what we are looking to achieve in this project. 