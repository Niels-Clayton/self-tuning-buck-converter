\chapter{Implementation}\label{C:implementation}


\section{PWM Generation}\label{S:pwm_gen_impl}

go ask Daniel about this because currently all I can think to say is about the bootstrap circuit and the code. But the bootstrap circuit is not important to the design, and was inly used due to covid since I will be using a gate driver in the final design. 

\subsubsection{TODO Gate driving and Bootstrapping} 

Discuss how the micro will be unable to directly drive the gate of the switching MOSFET, as it's $V_{GS_{on}}$ will be too large. Because of this a gate driver will need to be selected. The driver will need to be driven using the 3.3V logic from the ESP32, and will need timings fast enough to function correctly at 100kHz. I can also discuss how during the lock-down I was unable to purchase a gate driver, and so I built a bootstrapping circuit from components I located around the house. 


\section{Inductor Current Peak to Peak Sensor}\label{S:current_sense_impl}

Do I just show a photo of the breadboard and say that I built it on that?  


\section{Control System}\label{S:control_impl}


\section{Software Architecture}

Discuss the ESPidf HAL (Hardware Abstraction Layer), and how it provides greater control of the system resources than the more commonly used arduino platform. 
\\
Then discuss how the system operates within the freeRTOS real time operating system, what allows for easy multitasking between the different time sensitive control loops that will have to be run on the micro.  
\\
Finally discuss the simple to use API that was implemented, which aims to abstract away the HAL layer. This API is implemented using collection of single header libraries written in the 'C' programming language, with each implementing only the core functionality of the hardware they are interacting with. This makes the software platform robust and simple to move to different embedded platforms, with such a move only requiring the re-implementation of the core functions of each library. 